\documentclass{scrartcl}% siehe <http://www.komascript.de>
% packages ---------------------------------------------------------------------------------------------------- packages
\usepackage[margin=2.5cm]{geometry}
\usepackage[utf8]{inputenc}
\usepackage[T1]{fontenc}
\usepackage[ngerman]{babel}
\usepackage{lmodern}
\usepackage{amsmath}
\usepackage{amssymb}
\usepackage{amsfonts}
\usepackage{cancel}
\usepackage{listings}
\usepackage{multirow}
\usepackage{hhline}
\usepackage{tikz}
\usepackage{pgfplots}
\usepackage[all,dvips,arc,curve,color,frame]{xy}
\usepackage[yyyymmdd,hhmm]{datetime}
\usepackage{float}
\usepackage{hyperref}
\usepackage[table]{xcolor,colortbl}
\usepackage{graphicx}
\usepackage{subcaption}
\usepackage{tabularx}
\usepackage{footnote}
\usepackage{multicol}
\usepackage{stackengine}
\usepackage[norndcorners,customcolors,shade]{hf-tikz}
\usepackage{siunitx}
\usepackage{algorithm}
\usepackage[]{algorithmic}
% \usepackage[noend]{algorithmic}

% pgf plotsets -------------------------------------------------------------------------------------------- pgf plotsets
\pgfplotsset{ticks=none}
\pgfplotsset{width=10cm, compat=1.5.1}

% tikz libraries ---------------------------------------------------------------------------------------- tikz libraries
\usetikzlibrary{shapes}
\usetikzlibrary{positioning}
\usetikzlibrary{shapes.misc}
\usetikzlibrary{shapes.arrows,calc,quotes,babel}
\usetikzlibrary{positioning,arrows,chains,scopes,fit}
\usetikzlibrary{matrix,backgrounds}
\usetikzlibrary{trees}
\usetikzlibrary{calc}
\usetikzlibrary{tikzmark}
\usetikzlibrary{intersections}


\sisetup{
locale = DE ,
per-mode = symbol
}

\algsetup{indent=2em}
\newlength\myindent
\setlength\myindent{2em}
\newcommand\bindent{%
    \begingroup
    \setlength{\itemindent}{\myindent}
    \addtolength{\algorithmicindent}{\myindent}
}
\newcommand\eindent{\endgroup}

% commands ---------------------------------------------------------------------------------------------------- commands
\setlength{\parindent}{0pt}
\newcommand{\includepic}[2]{\includegraphics[width=#2\textwidth,height=#2\textheight,keepaspectratio]{#1}}
\newcommand\circlearound[1]{%
\tikz[baseline]\node[draw,shape=circle,anchor=base] {#1} ;}
\newcommand{\emptyrow}[0]{\multicolumn{1}{c}{} & \multicolumn{1}{c}{} & \multicolumn{1}{c}{} & \multicolumn{1}{c}{} &
\multicolumn{1}{c}{} & \multicolumn{1}{c}{} & \multicolumn{1}{c}{} & \multicolumn{1}{c}{} &
\multicolumn{1}{c}{} & \multicolumn{1}{c|}{} & & \\}
\newcommand\underbracetext[1]{\raisebox{1ex}{\ensuremath{\underbrace{\hphantom{\text{#1}}}_{\text{\normalsize #1}}}}}
\newcommand{\headerline}[3]{
\subject{#2}
\title{#1}
\subtitle{#3}
\date{Letztes Update: \today \ - \currenttime \ Uhr}
\maketitle
}
\newcommand{\newproof}[2]{
\vspace*{0.3cm} \textbf{\textsf{Satz:}} #1

\vspace*{0.3cm} \textbf{\textsf{Beweis:}} #2 \vspace*{0.3cm}
}
%\newcommand{\uparrow}[0]{\rlap{\scalebox{1.6}{$\uparrow$}}}
\newlength\dlf
\newcommand{\proofend}[0]{\hfill $\square$}
\newcommand\drawbi[7][4pt]{%
    \path(#2)--(#3)coordinate[at start](h1)coordinate[at end](h2);
    \draw[#4]($(h1)!#1!90:(h2)$)-- node [auto=left] {#5} ($(h2)!#1!-90:(h1)$);
    \draw[#6]($(h1)!#1!-90:(h2)$)-- node [auto=right] {#7} ($(h2)!#1!90:(h1)$);
    }

% colors -------------------------------------------------------------------------------------------------------- colors
\definecolor{amethyst}{rgb}{0.6, 0.4, 0.8}
\definecolor{anti-flashwhite}{rgb}{0.95, 0.95, 0.96}
\definecolor{applegreen}{rgb}{0.55, 0.71, 0.0}
\definecolor{babyblue}{rgb}{0.54, 0.81, 0.94}
\definecolor{bistre}{rgb}{0.24, 0.17, 0.12}
\definecolor{buff}{rgb}{0.94, 0.86, 0.51}
\definecolor{dodgerblue}{rgb}{0.12, 0.56, 1.0}
\definecolor{icterine}{rgb}{0.99, 0.97, 0.37}
\definecolor{tan}{rgb}{0.82, 0.71, 0.55}
\definecolor{upforestgreen}{rgb}{0.0, 0.27, 0.13}

% listing colors
\definecolor{listingBlue}{rgb}{0.13,0.13,1}
\definecolor{listingGreen}{rgb}{0,0.5,0}
\definecolor{listingRed}{rgb}{0.9,0,0}
\definecolor{listingGrey}{rgb}{0.46,0.45,0.48}

% listings ---------------------------------------------------------------------------------------------------- listings
\lstset{basicstyle=\ttfamily}
\lstset{literate=%
{Ö}{{\"O}}1
{Ä}{{\"A}}1
{Ü}{{\"U}}1
{ß}{{\ss}}1
{ü}{{\"u}}1
{ä}{{\"a}}1
{ö}{{\"o}}1
}
\lstset{language=Java,
showspaces=false,
showtabs=false,
breaklines=true,
showstringspaces=false,
breakatwhitespace=true,
commentstyle=\color{listingGreen},
keywordstyle=\color{listingBlue},
stringstyle=\color{listingRed},
basicstyle=\ttfamily,
moredelim=[il][\textcolor{listingGrey}]{$$},
moredelim=[is][\textcolor{listingGrey}]{\%\%}{\%\%}
}
\lstdefinelanguage{Kotlin}{
  keywords={package, as, typealias, this, super, val, var, fun, for, null, true, false, is, in, throw, return, break, continue, object, if, try, else, while, do, when, yield, typeof, yield, typeof, class, interface, enum, object, override, public, private, get, set, import, abstract, },
  keywordstyle=\color{listingBlue}\bfseries,
  ndkeywords={@Deprecated, Iterable, Int, Integer, Float, Double, String, Runnable, dynamic},
  ndkeywordstyle=\color{amethyst}\bfseries,
  emph={println, return@, forEach,},
  emphstyle={\color{orange}},
  identifierstyle=\color{black},
  sensitive=true,
  commentstyle=\color{listingGreen}\ttfamily,
  comment=[l]{//},
  morecomment=[s]{/*}{*/},
  stringstyle=\color{listingRed}\ttfamily,
  morestring=[b]",
  morestring=[s]{"""*}{*"""},
  numbers=left,
}

% misc ------------------------------------------------------------------------------------------------------------ misc
\graphicspath{ {../lectures-img/} }
\setlength{\parindent}{0pt}

\newcolumntype{C}[1]{>{\centering\arraybackslash}p{#1}} % zentriert mit Breitenangabe
\newcolumntype{L}[1]{>{\raggedright\arraybackslash}p{#1}} % rechtsbündig mit Breitenangabe
\newcolumntype{R}[1]{>{\raggedleft\arraybackslash}p{#1}} % rechtsbündig mit Breitenangabe


\definecolor{cadmiumred}{rgb}{0.89, 0.0, 0.13}
\definecolor{cadmiumorange}{rgb}{0.93, 0.53, 0.18}
\definecolor{cadmiumyellow}{rgb}{1.0, 0.96, 0.0}
\definecolor{cadmiumgreen}{rgb}{0.0, 0.42, 0.24}
\definecolor{blizzardblue}{rgb}{0.67, 0.9, 0.93}
\definecolor{persianblue}{rgb}{0.11, 0.22, 0.73}
\definecolor{darkviolet}{rgb}{0.58, 0.0, 0.83}

\begin{document}
    \headerline{Algorithmen und Berechenbarkeit}{Vorlesungsmitschrift}{Vorlesung 12}

    \section*{Beispiele für Turingmaschinen}
    \subsection*{Beispiel 1}
    Gegeben sei die Sprache $L_1$, die aus allen Wörtern über \texttt{0} und \texttt{1} besteht,
    und deren Wörter auf \texttt{0} enden:
    \begin{equation*}
        L = \{w0 \ | \ w \in \{0,1\}^* \}
    \end{equation*}

    Eine TM, die $L_1$ entscheidet, könnte folgendermaßen aussehen:
    \begin{equation*}
        M_1 = (\{q_0, q_1,\bar{q} \},\{0,1\},\{0,1,\text{B}\}, \text{B}, q_1, \bar{q}, \delta)
    \end{equation*}
    Die Übergangsfunktionen sind dabei

    \begin{figure}[htb]
        \centering
        \begin{tabular}{r|l|l|l}
            \textbf{\textsf{$\delta$}} & \texttt{0} & \texttt{1} & \texttt{B} \\
            \hline
            $q_0$ & \underline{$\{q_0, 0, R\}$} & $\{q_1, 1, R\}$ & $\{\bar{q}, 1, N \}$ \\
            $q_1$ & $\{q_0, 0, R\}$ & $\{q_1, 1, R\}$ & $\{\bar{q}, 0, N \}$ \\
            \hline
        \end{tabular}
    \end{figure}

    Zu lesen ist zum Beispiel die unterstrichene Übergangsfunktion wie folgt:
    Man befindet sich in Zustand $q_0$ und ließt eine \texttt{0} ein,
    dann bleibt man in Zustand $q_0$, schreibt eine \texttt{0} an die Stelle,
    an der vorher schon eine \texttt{0} war, und bewegt den SLK nach rechts.

    \vspace*{0.3cm}
    Oftmals ist es ratsam, dass man sich klarmacht, was die Zustände bedeuten:
    \begin{itemize}
        \item [$\rightarrow$] $q_0$ bedeutet, man hat als letztes Zeichen eine \texttt{0} gelesen
        \item [$\rightarrow$] $q_1$ bedeutet, man hat als letztes Zeichen keine \texttt{0} gelesen
    \end{itemize}

    Für das Wort $w_1=\texttt{010101}$ ergibt sich der folgende Ablauf:
    \begin{equation*}
        \begin{align}
            \Rightarrow \quad & q_{1}010101 & \vdash \quad & 0q_{0}10101 & \vdash \quad & 01q_{1}0101 & \vdash \quad & 010q_{0}101\\
            \vdash \quad & 0101q_{1}01 & \vdash \quad & 01010q_{0}1 & \vdash \quad & 010101q_{1}B & \vdash \quad & 010101\bar{q}0\\
        \end{align}
    \end{equation*}
    Da der letzte Zustand erreicht ist und $w_1$ mit einer \texttt{1} endet, schreibt die TM in der letzten Konfiguration eine \texttt{0}.

    \newpage
    \subsection*{Beispiel 2}
    Gegeben sei die Sprache $L_2$:
    \begin{equation*}
        L_2 = \{0^n 1^n \ | \ n \geq 1 \}
    \end{equation*}

    Eine TM, die $L_2$ entscheidet, könnte folgendermaßen aussehen:
    \begin{equation*}
        M_2 = (\{q_0, q_1,q_2,q_3,q_4,q_5,q_6,\bar{q}\},\{0,1\},\{0,1,\text{B}\}, \text{B}, q_0, \bar{q}, \delta)
    \end{equation*}

    Die Übergangsfunktionen sind dabei

    \begin{figure}[htb]
        \centering
        \begin{tabular}{r|l|l|l}
            \textbf{\textsf{$\delta$}} & \texttt{0} & \texttt{1} & \texttt{B} \\
            \hline
            $q_0$ & $\{q_0, 0, R\}$ & $\{q_1, 1, R\}$ & $\{\bar{q}, 0, N \}$ \\
            $q_1$ & $\{\bar{q}, 0, N\}$ & $\{q_1, 1, R\}$ & $\{q_2, B, L \}$ \\
            \hline \hline
            $q_2$ & $\{\bar{q}, 0, N\}$ & $\{q_3, B, L\}$ & $\{\bar{q}, 0, N \}$ \\
            $q_3$ & $\{q_3, 0, L\}$ & $\{q_3, 1, L\}$ & $\{q_4, B, R \}$ \\
            $q_4$ & $\{q_5, B, R\}$ & $\{\bar{q}, 0, N\}$ & $\{\bar{q}, 0, N \}$ \\
            $q_5$ & $\{q_6, 0, R\}$ & $\{q_6, 1, R\}$ & $\{\bar{q}, 1, N \}$ \\
            $q_6$ & $\{q_6, 0, R\}$ & $\{q_6, 1, R\}$ & $\{q_2, B, L \}$ \\
        \end{tabular}
    \end{figure}
    \begin{itemize}
        \item [$\rightarrow$] $q_0$ bedeutet, man hat bis jetzt keine \texttt{1} gelesen
        \item [$\rightarrow$] $q_1$ bedeutet, man hat $\texttt{0}^i$ und danach mindestens eine \texttt{1} gelesen
        \item [$\rightarrow$] $q_2$ bedeutet, der SLK steht auf dem letzten Zeichen (also nicht \texttt{B})
        \item [$\rightarrow$] $q_3$ bedeutet, der SLK wird nach ganz links bewegt
        \item [$\rightarrow$] $q_4$ bedeutet, der SLK steht auf dem ersten Zeichen (also nicht \texttt{B})
        \item [$\rightarrow$] $q_5$ bedeutet, man überprüft, ob man schon fertig ist
        \item [$\rightarrow$] $q_6$ bedeutet, der SLK wird nach ganz rechts bewegt
    \end{itemize}

    Die Entscheidung, ob ein Wort in der Sprache liegt, läuft in zwei Phasen ab:
    \begin{enumerate}
        \item In dieser Phase überprüft man, ob die Eingabe der Form $0^{i}1^j$ mit $i \geq 0$, $j \geq 1$ entspricht
        \item Zu Beginn dieser Phase steht der SLK auf dem letzten Zeichen der Eingabe.
        In dieser zweiten Phase überprüft man durch abwechselndes Löschen der jeweils äußeren Zeichen, ob $i=j$.
    \end{enumerate}

    Beispielhaft wird die zweite Phase für das Wort $w_2=\texttt{000111}$ aufgezeigt:
    \begin{equation*}
        \begin{align}
            \Rightarrow \quad & 00011q_{2}1 & \vdash \quad & 0001q_{3}1B & \vdash \quad & \cdots & \vdash \quad & q_3B00011\\
            \vdash \quad & Bq_{4}00011 & \vdash \quad & Bq_{5}0011 & \vdash \quad & 0q_{5}011 & \vdash \quad & \cdots \\
            \vdash \quad & 0011q_{6}B & \vdash \quad & 001q_{2}1 & \vdash \quad & \cdots\ \cdots & \vdash \quad & Bq_{5}BBB \\
            \vdash \quad & \bar{q}1 \\
        \end{align}
    \end{equation*}
    Da der letzte Zustand erreicht ist und für $w_2$ gilt $i=j$, schreibt die TM in der letzten Konfiguration eine \texttt{1}.

    \section*{Programmierung einer TM}
    Die \textit{Programmierung} einer TM ist relativ mühsam,
    dennoch können alle Sprachkonstrukte aus \textit{normalen Programmiersprachen} dargestellt werden:
    \begin{itemize}
        \item {Eine boolsche Variable im Programm kann folgendermaßen in den Zustandsraum kodiert werden
        \begin{equation*}
            \begin{align}
                \text{Zustände alt:} \quad & Q \\
                \text{Zustände neu:} \quad & Q \times \{0,1\}
            \end{align}
        \end{equation*}
        }
        \item {Eine Variable $x$ mit konstant vielen Zuständen $\{0,1,2,3,\cdots, k \}$ kann ebenfalls in den Zustandsraum kodiert werden.
        \begin{equation*}
            Q_{\text{neu}} = \{Q \times \{0,1,2,3,\cdots, k \} \}
        \end{equation*}
        }
        \item {Man kann mit einer einspurigen TM auch eine $k$-spuringe TM simulieren mit jeweils einem Bandalphabet für ein $k$-Tupel.
        \begin{figure}[htb]
            \centering
            \begin{tabular}{C{1em}|C{1em}|C{1em}|C{1em}|C{1em}|C{1em}|C{1em}|C{1em}|C{1em}|C{1em}|C{1em}|C{1em}|C{1em}|C{1em}}
                \hline
                & & & & & & & & & & & & & \\ \hline
                & & & & & & & & & & & & & \\ \hline
                & & & & & & & & & & & & & \\ \hline
                & & & & & & & & & & & & & \\ \hline
            \end{tabular}
        \end{figure}
        \begin{figure}[H]
            \centering
            \begin{subfigure}{0.3\textwidth}
                \raggedleft
                \begin{equation*}
                    \begin{bmatrix}
                        0 \\
                        1
                    \end{bmatrix}
                    \begin{bmatrix}
                        1 \\
                        1
                    \end{bmatrix}
                \end{equation*}
            \end{subfigure}%
            \raggedright
            \begin{subfigure}{.7\textwidth}
                $\Rightarrow$ Neues Bandalphabt $|\Gamma_{\text{neu}} | = |\Gamma_{\text{alt}}|^n$
            \end{subfigure}
        \end{figure}
        Das ist zum Beispiel dann nützlich, wenn zwei Binärzahlen miteinander addiert werden.
        Anstatt auf bin$(a)\#$bin$(b)$ zu operieren,
        kopiert man bin$(b)$ auf die zweite Spur und berechnet in der dritten Spur das Ergebnis.
        }
        \item Konstant viele Variablen mit nicht konstantem Wertebereich $\rightarrow$ jeweils eine Spur pro Variable
        \item Unterprogramm und Rekursion $\rightarrow$ Reservierung einer Spur pro Prozeduraufruf
        \item $\cdots$
    \end{itemize}

    \newpage
    \section*{Mehrband TM}
    Eine Mehrband TM hat für jede Spur einen unabhängigen SLK. Die Zustandsübergänge haben die Form
    \begin{equation*}
        \delta : ( Q \textbackslash \{ \bar{q}\} \times \Gamma^{k} \rightarrow Q \times \Gamma^{k} \times \{ L,R,N \}^k)
    \end{equationn*}

    Das erste Band enthält die Ein- bzw. Ausgabe, alle anderen Bänder sind initial mit \texttt{B}s gefüllt.

    \vspace*{0.3cm}
    \textbf{\textsf{Satz:}} Eine $k$-Band TM $\mathcal{M}$,
    die mit einer Rechenzeit von $f(n)$ und einem Platzverbrauch von $s(n)$ auskommt,
    kann von einer \texttt{1}-Band TM $\mathcal{M}'$ mit der Rechenzeit von $\mathcal{O}(f^2(n))$ und einem Platzbedarf von $\mathcal{O}(s(n))$ simuliert werden.

     \vspace*{0.3cm}
    \textbf{\textsf{Beweisidee:}} Man simuliert eine $k$-Band TM durch eine $2k$-Band TM. Zum Beispiel wird die folgende $2$-Band TM
    \begin{figure}[htb]
        \centering
        \begin{tabular}{C{1em}|C{1em}|C{1em}|C{1em}|C{1em}|C{1em}|C{1em}|C{1em}|C{1em}|C{1em}|C{1em}}
            \hline
            $\cdots$ & \texttt{B} & \texttt{0} & \texttt{1} & \texttt{2} & \texttt{1} & \texttt{2} & \texttt{1} & \texttt{B} & \texttt{B} & $\cdots$\\
            $\cdots$ & \texttt{B} & \texttt{B} & \texttt{1} & \texttt{1} & \texttt{3} & \texttt{2} & \texttt{B} & \texttt{B} & \texttt{B} & $\cdots$\\
            \hline
        \end{tabular}
    \end{figure}

    durch
    \begin{figure}[htb]
        \centering
        \begin{tabular}{C{1em}|C{1em}|C{1em}|C{1em}|C{1em}|C{1em}|C{1em}|C{1em}|C{1em}|C{1em}|C{1em}}
            \hline
            $\cdots$ & \texttt{B} & \texttt{0} & \texttt{1} & \texttt{2} & \texttt{1} & \texttt{2} & \texttt{1} & \texttt{B} & \texttt{B} & $\cdots$\\
            $\cdots$ & \texttt{B} & \texttt{B} & \texttt{B} & \texttt{$\#$} & \texttt{B} & \texttt{B} & \texttt{B} & \texttt{B} & \texttt{B} & $\cdots$\\
            $\cdots$ & \texttt{B} & \texttt{B} & \texttt{1} & \texttt{1} & \texttt{3} & \texttt{2} & \texttt{B} & \texttt{B} & \texttt{B} & $\cdots$\\
            $\cdots$ & \texttt{B} & \texttt{B} & \texttt{B} & \texttt{B} & \texttt{B} & \texttt{B} & \texttt{$\#$} & \texttt{B} & \texttt{B} & $\cdots$\\
            \hline
        \end{tabular}
    \end{figure}

    simuliert.

    Für einen Rechenschritt der $k$-Band TM wird über den beschriebenen Teil der $2k$-Band TM gelaufen
    und die Zeichen unter den SLK der $k$-Bänder werden eingesammelt (und im Zustand gespeichert).
    In einem zweiten Durchlauf über die Bänder werden die Zeichen und Kopfpositionen upgedatet.

    \vspace*{0.3cm}
    Anzahl beschriebener Zellen $\leq f(n)$
    \begin{itemize}
        \item [$\Rightarrow$] Ein Schritt der $k$-Band TM kostet $\mathcal{O}(f(n))$ Schritte auf der $2k$-Band TM
        \item [$\Rightarrow$] $\mathcal{O}(f^2(n))$
    \end{itemize}

\end{document}