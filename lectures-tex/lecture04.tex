\documentclass{scrartcl}% siehe <http://www.komascript.de>
\usepackage[margin=2.5cm]{geometry}
\usepackage[utf8]{inputenc}
\usepackage[T1]{fontenc}
\usepackage[ngerman]{babel}
\usepackage{lmodern}
\usepackage{hyperref}
\usepackage{amsmath}
\usepackage{amssymb}
\usepackage{cancel}
\usepackage{mathrsfs}
\usepackage{mathtools}
\usepackage{listings}

\lstset{basicstyle=\ttfamily}
\lstset{literate=%
{Ö}{{\"O}}1
{Ä}{{\"A}}1
{Ü}{{\"U}}1
{ß}{{\ss}}1
{ü}{{\"u}}1
{ä}{{\"a}}1
{ö}{{\"o}}1
}

\newtheorem{theorem}{Theorem}

\newcommand{\includepic}[2]{\includegraphics[width=#2\textwidth,height=#2\textheight,keepaspectratio]{#1}}

\usepackage[yyyymmdd,hhmm]{datetime}

% Base path of images
\usepackage{graphicx}
\graphicspath{ {../lectures-img/} }

\begin{document}
    % ----------------------------------------------------------------------------
    \subject{Vorlesungsmitschrieb}
    \title{Algorithmen und Berechenbarkeit}
    \subtitle{Vorlesung 04}
    \date{Letztes Update: \today \ - \currenttime \ Uhr}
    \maketitle
    % ----------------------------------------------------------------------------


    \subsection*{Las-Vegas-Algorithmus: Randomisierter Quicksort-Algorithmus}
    \label{subsec:las-vegas-algorithmus:RandomisierterQuicksort-algorithmus}

    \begin{lstlisting}
        Eingabe A: [1, 2, 3, ... , n]
            Wähle p aus Eingabe zufällig gleichverteilt
            Füge p an der richtigen Stelle ein
    \end{lstlisting}

    \begin{figure}[htb]
        \centering
        \includepic{lec_04_a}{0.45}
        \caption{Einfügen eines Elements an der \textit{richtigen} Stelle}
    \end{figure}
    Der Algorithmus wird dann für die linke Seite $RQS(A_l)$ bzw. $RQS(A_r)$ rekursiv aufgerufen.
    Um $p$ \textit{zufällig gleichverteilt} aufzurufen, können verschiedene Ansätze gewählt werden.
    Zum Beispiel nimmt man immer das mittige Element ($\frac{n}{2}$)
    oder man berechnet den Median (das klappt in $O(n)$) und wählt das entsprechende Element aus.
    Beide Fälle sind suboptimal.

    \subsubsection*{Versuch einer Analyse}
    Zuerst stellt man sich die folgende \textbf{\textsf{Frage:}} Was ist die erwartete Länge von $A_l$ bzw. $A_r$?

    \begin{equation*}
        E(|A_l|) = E(|A_r|)= \sum^{n-1}_{i = 0}\frac{1}{n} \cdot i = \frac{1}{n} \cdot \frac{(n-1) \cdot n}{2} = \frac{n -1}{2}
    \end{equation*}
    Daraus folgt, dass die Tiefe des Rekursionsbaums  $\mathcal{O}(\log(n))$ sein muss.

    \begin{figure}[htb]
        \centering
        \includepic{lec_04_b}{0.45}
        \caption{Die Gesamtlaufzeit ergibt sich somit als $\mathcal{O}(n \cdot \log(n))$}
    \end{figure}

    Dieser Ansatz der Analyse kann nicht richtig sein,
    denn folgender Quicksort-Algorithmus hätte mit derselben Folgerung dieselbe Laufzeit, was aber nicht stimmt.

    \begin{lstlisting}
        Eingabe A: [1, 2, 3, ... , n]
            Wähle p
                mit 50% Wahrscheinlichkeit als kleinstes Element
                mit 50% Wahrscheinlichkeit als größtes Element
            Füge p an der richtigen Stelle ein
    \end{lstlisting}

    Man wählt $p$ also entweder als erstes (kleinstes) oder letztes (größtes) Element, und startet den Algorithmus rekursiv erneut.
    Nach obiger Rechnung ergäbe sich auch hier wieder eine Länge von

    \begin{equation*}
        E(|A_l|) = E(|A_r|) = \frac{1}{2} \cdot 0 + \frac{1}{2}(n-1) = \frac{n-1}{2}
    \end{equation*}

    Da bei einem rekursiven Aufruf aber jeweils wieder $n-1$ Elemente in der Eingabe enthalten sind, kann das also nicht stimmen.

    \subsubsection*{Korrekte Analyse des Random Quicksort-Algorithmus}
    Wir gehen zunächst von folgender \textbf{Annahme} aus:
    Die zu sortierenden Elemente sind $s_1, s_2, s_3, ..., s_n$ und außerdem gilt $s_1 < s_2 < s_3 < ... < s_n$.
    Die Zufallsvariable wird festgelegt als

    \begin{equation*}X_{ij}=
        \[ \begin{cases}
               1 & \text{falls während des Alorithmus } s_i \text{ mit } s_j \text{ verglichen} \\
               0 & \text{sonst}
        \end{cases}
        \]
    \end{equation*}

    Die Gesamtlaufzeit des Random Quicksort-Algorithmus ist damit

    \begin{equation*}
        \begin{flalign}
            \sum_{i<j}X_{ij}        &= \sum_{i=1}^{n-1} \cdot \sum_{j=i+1}^{n} X_{ij}&&\\\nonumber
            E\left(\sum_{i<j}X_{ij}\right)  &= \sum_{i<j}E(X_{ij}) \text{\quad (Linearität des Erwartungswerts)}&&\\\nonumber
                                            &= \sum_{i<j}(0 \cdot P(X_{ij} = 0) + 1 \cdot P(X_{ij} = 1))\\\nonumber
                                            &= \sum_{i<j}\underbrace{P(X_{ij}=1)}_{P_{ij}}
        \end{flalign}
    \end{equation*}

    Als Nächstes soll nun $P_{ij}$ berechnet werden, wobei folgende Überlegungen zu beachten sind:
    \begin{itemize}
        \item Elemente mit großer Rangdifferenz werden nur mit geringer Wahrscheinlichkeit miteinander verglichen,
                da viele Chancen bestehen, dass sie im Rekursionsbaum getrennt werden.
        \item Elemente mit kleiner Rangdifferenz werden mit größerer Wahrschenlichkeit miteinander verglichen.
        \item Elemente mit der Rangdifferenz von $1$ müssen miteinander verglichen werden.
    \end{itemize}
    \newpage

    In folgendem Ablauf des Random Quicksorts entspricht ein Knoten einem Pivotelement im entsprechenden Aufruf.
    \begin{figure}[htb]
        \centering
        \includepic{lec_04_c}{0.45}
        \caption{Die erzeugte Permutation ist $\pi = s_{70}, s_{30}, s_{90}, s_{11}, s_{37}, s_{76}, s_{106} $}
    \end{figure}

    \textit{Anmerkung: $\pi$ ist nicht zufällig gleichverteilt, da manche Permutationen gar nicht vorkommen können.}
    \newline
    Als Nächstes stellt sich die \textbf{\textsf{Frage}}, wie es sich auf die Laufzeit auswirkt, wenn $s_i$ nicht verglichen wird.

    Dann taucht ein Element $s_k$ mit $s_i < s_k < s_j$ als Pivotelement vor $s_i$ und vor $s_j$ in $\pi$ auf,
    denn $s_i$ wird genau dann mit $s_j$ verglichen, wenn $s_i$ oder $s_j$ das erste Element aus $s_i, s_{i+1}, ... , s_j$ in $\pi$ ist.
    Außerdem gilt: Mit gleicher Wahrscheinlichkeit ist jedes der Elemente $s_i, s_{i+1}, ... , s_j$ das erste, das in $\pi$ auftaucht.

    Die Wahrscheinlichkeit, dass $s_i$ mit $s_j$ verglichen wird, beträgt also
    \begin{equation*}
        P_{ij}:= P(s_i \text{ wird mit } s_j \text{ verglichen}) = \frac{2}{j-i+1}
    \end{equation*}

    Damit kann nun die Laufzeit eingeordnet werden:
%\sum_{i<j}X_{ij}        &= \sum_{i=1}^{n-1} \cdot \sum_{j=i+1}^{n} X_{ij}&&\\\nonumber
    \begin{equation*}
        \begin{flalign}
            E\left(\sum_{r<j}X_{ij}\right)  &= \sum_{i+1}^{n-1} \cdot \sum_{j=i+1}^{n}E(X_{ij})&&\\\nonumber
                                            &= \sum_{i+1}^{n-1} \cdot \sum_{j=i+1}^{n}P_{ij}&&\\\nonumber
                                            &= \underbrace{\sum_{i+1}^{n-1} \cdot \sum_{j=i+1}^{n}\frac{2}{j-i+1}}_{\frac{2}{2}+\frac{2}{3}+\frac{2}{4}+\frac{2}{5}+\ ...}&&\\\nonumber
                                            &\leq \sum_{i=1}^{n-1} \cdot \sum_{l=2}^{n}\frac{2}{l}&&\\\nonumber
                                            &= \sum_{i=1}^{n-1}2 \cdot \sum_{l=2}^{n} \frac{1}{l}&&\\\nonumber
                                            &= \sum_{i=1}^{n-1}2 \cdot \mathcal{O}(H_n)&&\\\nonumber
                                            &= \sum_{i=1}^{n-1}2 \cdot \mathcal{O}(\log(n))&&\\\nonumber
                                            &= \mathcal{O}(n \cdot \log(n))
        \end{flalign}
    \end{equation*}
    \textit{Anmerkung: $H_n$ beschreibt die $n-$te harmonische Zahl.}

    \section*{Zero-Knowledge-Proof}
    \label{sec:zero-knowledge-proof}
    Zunächst soll eine \textbf{\textsf{Anwendung}} des "`Zero-Knowledge-Proofs"' beschrieben werden:
    Wie könnte eine Bank beim Onlinebanking beweisen, dass sie weiterhin die Bank ist (die betrachtete Website weiterhin zur Bank gehört)?

    Ein möglicher \textbf{\textsf{Ansatz}} besagt, dass die Bank (im Folgenden auch einfach \textbf{\textsf{Alice}})
    ihrem Kunden (\textbf{\textsf{Bob}}) beweisen muss, dass sie ein Geheimnis kennt, welches nur Alice bekannt sein kann,
    allerdings ohne das Geheimnis zu verraten.

    Außerdem sollen die folgenden \textbf{\textsf{Eigenschaften}} erfüllt sein:

    \begin{itemize}
        \item Falls Alice das Geheimnis nicht kennt, wird das mit hoher Wahrscheinlichkeit von Bob erkannt.
        \item Alice gibt nichts über das Geheimnis preis, was Bob nicht alleine rausfinden könnte.
    \end{itemize}

    Das Protokoll basiert auf dem Graphisomorphie-Problem (Isomorphismus zwischen zwei öffentlichen Graphen $G_1$ und $G_2$).

    Alice kann $G_1$ und $G_2$ wie folgt erzeugen: Sie wählt $G_1$ zufällig und außerdem eine zufällige Permutation $\Phi$ der Knoten von $G_1$, was dann $G_2$ definiert.
    $\Phi$ bleibt geheim, wohingegen $G_1$ und $G_2$ veröffentlicht werden.
    \newline
    \newline
    \subsection*{Protokoll, durch welches Alice Bob überzeugt, P zu kennen:}
    \newline
    \newline
    Alice permutiert $G_{j}$ wobei $j \in \{1,2\}$ mit einer zufälligen Permutation $\pi$ zu einem Graphen $H$.
    Dabei stellt sie sicher, dass $H\neq G_1$ und $H\neq G_2$.
    Außerdem sendet sie $H$ an Bob.

    Bob möchte sich davon überzeugen, dass Alice die Isomorphie zwischen $G_1$ und $G_2$ kennt,
    indem er $k \in \{1,2\}$ zufällig wählt und Alice bittet, die Isomorphie zwischen $G_k$ und $H$ offenzulegen.

    Falls nun Alice das Geheimnis $\Phi$ kennt, so kann sie \textbf{\textsf{immer}} $\Phi$ bzw. $\Phi \circ \pi$ zurückgeben.
    Falls Alice $\Phi$ nicht kennt und Bob den Graphen wählt, aus dem Alice $H$ generiert hat,
    kann sie leicht antworten, falls jedoch Bob den anderen Graphen gewählt hat,
    müsste sie das Graphisomorphie-Problem lösen, um die korrekte Antwort zu geben.

    Bei $x-$maliger Wiederholung ist die Wahrscheinlichkeit, dass Bob immer den richtigen Graphen erwischt, aus dem $H$ erzeugt wurde
    \begin{equation*}
        \frac{1}{2^x}
    \end{equation*}
    Daraus folgt, dass die Wahrscheinlichkeit, dass Bob erkennt, dass Alice $\Phi$ nicht kennt,
    \begin{equation*}
        1-\frac{1}{2^x}
    \end{equation*}
    beträgt.

    \begin{theorem}
        Alice verrät nichts über $\Phi$.
    \end{theorem}

    \begin{proof}[\textbf{Beweis}]
        Bob lernt Isomorphie zwischen z.B. $G_1$ und einer zufälligen Permutation von $G_1$.
        Das hätte er auch selber herausfinden können.
    \end{proof}

%    \hrulefill

    \newpage
    \section*{Anhang}
    \label{sec:anhang}

    \subsubsection*{Graphisomorphie}

    Gegeben seien zwei Graphen $G_1(V_1, E_1)$ und $G_2(V_2, E_2)$.
    Graphisomorphie bedeutet, es gibt eine bijektive Funktion $\phi$, sodass

    \begin{equation*}
        \forall a = (v,w) \in E_1 \Leftrightarrow (\phi(v), \phi(w)) \in E_2
    \end{equation*}

    \subsubsection*{Graphisomorphie-Problem}

    Gegeben seien zwei Graphen $G_3(V_3, E_3)$ und $G_4(V_4, E_4)$ und man stellt sich die \textbf{\textsf{Frage}}:
    Wann sind $G_1$ und $G_2$ isomorph?

    \begin{figure}[htb]
        \centering
        \includepic{lec_04_d}{0.8}
        \caption{Die Graphen $G_3$ und $G_4$}
    \end{figure}

    Die folgende Tabelle zeigt die Isomorphie zwischen $v \in V_3$ und $\phi(v)$.
    \begin{table}[!ht]
        \centering
        \begin{tabular}{lll}
            $v \in V_3$ & \phi(v) & \Phi(v)\\
            \hline
            $1$ & $B$ & $B$ \\
            $2$ & $D$ & $A$ \\
            $3$ & $C$ & $D$ \\
            $4$ & $A$ & $C$ \\
            \hline \\
        \end{tabular}
    \end{table}

    $v \in V_3$ und $\Phi(v)$ sind nicht isomorph.
    Festgestellt werden kann dies zum Beispiel über den Knotengrad oder durch die folgende Erkenntnis:
    \begin{equation*}
        \begin{flalign}
            2,4 \in G_3&&\\\nonumber
            A,C \notin G_4&&
        \end{flalign}
    \end{equation*}

    \newpage
    Es folgen weitere Beispiele:
    \begin{figure}[htb]
        \centering
        \includepic{lec_04_e}{0.8}
        \caption{Die nicht isomorphen Graphen $G_5$ und $G_6$}
    \end{figure}
    \begin{figure}[htb]
        \centering
        \includepic{lec_04_f}{0.8}
        \caption{Die isomorphen Graphen $G_7$ und $G_8$}
    \end{figure}
    \begin{figure}[htb]
        \centering
        \includepic{lec_04_g}{0.8}
        \caption{Die nicht isomorphen Graphen $G_9$ und $G_{10}$ ($D$ hat Grad $4$, $2$ hat Grad $3$)}
    \end{figure}

    \textit{Anmerkung: Bis zum Jahr 2015 dachten Informatiker, Graphenisomorphie läge in \textbf{\textsf{NP-schwer}}.
    Im Jahr 2015 wurde jedoch gezeigt, dass das Problem der Graphenisomorphie in Quasipolynomzeit lösbar ist:}
    \begin{equation*}
        \mathcal{O}(2^{O(\log(n))^2})
    \end{equation*}

\end{document}