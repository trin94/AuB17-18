\documentclass{scrartcl}% siehe <http://www.komascript.de>
% packages ---------------------------------------------------------------------------------------------------- packages
\usepackage[margin=2.5cm]{geometry}
\usepackage[utf8]{inputenc}
\usepackage[T1]{fontenc}
\usepackage[ngerman]{babel}
\usepackage{lmodern}
\usepackage{amsmath}
\usepackage{amssymb}
\usepackage{amsfonts}
\usepackage{cancel}
\usepackage{listings}
\usepackage{multirow}
\usepackage{hhline}
\usepackage{tikz}
\usepackage{pgfplots}
\usepackage[all,dvips,arc,curve,color,frame]{xy}
\usepackage[yyyymmdd,hhmm]{datetime}
\usepackage{float}
\usepackage{hyperref}
\usepackage[table]{xcolor,colortbl}
\usepackage{graphicx}
\usepackage{subcaption}
\usepackage{tabularx}
\usepackage{footnote}
\usepackage{multicol}
\usepackage{stackengine}
\usepackage[norndcorners,customcolors,shade]{hf-tikz}
\usepackage{siunitx}
\usepackage{algorithm}
\usepackage[]{algorithmic}
% \usepackage[noend]{algorithmic}

% pgf plotsets -------------------------------------------------------------------------------------------- pgf plotsets
\pgfplotsset{ticks=none}
\pgfplotsset{width=10cm, compat=1.5.1}

% tikz libraries ---------------------------------------------------------------------------------------- tikz libraries
\usetikzlibrary{shapes}
\usetikzlibrary{positioning}
\usetikzlibrary{shapes.misc}
\usetikzlibrary{shapes.arrows,calc,quotes,babel}
\usetikzlibrary{positioning,arrows,chains,scopes,fit}
\usetikzlibrary{matrix,backgrounds}
\usetikzlibrary{trees}
\usetikzlibrary{calc}
\usetikzlibrary{tikzmark}
\usetikzlibrary{intersections}


\sisetup{
locale = DE ,
per-mode = symbol
}

\algsetup{indent=2em}
\newlength\myindent
\setlength\myindent{2em}
\newcommand\bindent{%
    \begingroup
    \setlength{\itemindent}{\myindent}
    \addtolength{\algorithmicindent}{\myindent}
}
\newcommand\eindent{\endgroup}

% commands ---------------------------------------------------------------------------------------------------- commands
\setlength{\parindent}{0pt}
\newcommand{\includepic}[2]{\includegraphics[width=#2\textwidth,height=#2\textheight,keepaspectratio]{#1}}
\newcommand\circlearound[1]{%
\tikz[baseline]\node[draw,shape=circle,anchor=base] {#1} ;}
\newcommand{\emptyrow}[0]{\multicolumn{1}{c}{} & \multicolumn{1}{c}{} & \multicolumn{1}{c}{} & \multicolumn{1}{c}{} &
\multicolumn{1}{c}{} & \multicolumn{1}{c}{} & \multicolumn{1}{c}{} & \multicolumn{1}{c}{} &
\multicolumn{1}{c}{} & \multicolumn{1}{c|}{} & & \\}
\newcommand\underbracetext[1]{\raisebox{1ex}{\ensuremath{\underbrace{\hphantom{\text{#1}}}_{\text{\normalsize #1}}}}}
\newcommand{\headerline}[3]{
\subject{#2}
\title{#1}
\subtitle{#3}
\date{Letztes Update: \today \ - \currenttime \ Uhr}
\maketitle
}
\newcommand{\newproof}[2]{
\vspace*{0.3cm} \textbf{\textsf{Satz:}} #1

\vspace*{0.3cm} \textbf{\textsf{Beweis:}} #2 \vspace*{0.3cm}
}
%\newcommand{\uparrow}[0]{\rlap{\scalebox{1.6}{$\uparrow$}}}
\newlength\dlf
\newcommand{\proofend}[0]{\hfill $\square$}
\newcommand\drawbi[7][4pt]{%
    \path(#2)--(#3)coordinate[at start](h1)coordinate[at end](h2);
    \draw[#4]($(h1)!#1!90:(h2)$)-- node [auto=left] {#5} ($(h2)!#1!-90:(h1)$);
    \draw[#6]($(h1)!#1!-90:(h2)$)-- node [auto=right] {#7} ($(h2)!#1!90:(h1)$);
    }

% colors -------------------------------------------------------------------------------------------------------- colors
\definecolor{amethyst}{rgb}{0.6, 0.4, 0.8}
\definecolor{anti-flashwhite}{rgb}{0.95, 0.95, 0.96}
\definecolor{applegreen}{rgb}{0.55, 0.71, 0.0}
\definecolor{babyblue}{rgb}{0.54, 0.81, 0.94}
\definecolor{bistre}{rgb}{0.24, 0.17, 0.12}
\definecolor{buff}{rgb}{0.94, 0.86, 0.51}
\definecolor{dodgerblue}{rgb}{0.12, 0.56, 1.0}
\definecolor{icterine}{rgb}{0.99, 0.97, 0.37}
\definecolor{tan}{rgb}{0.82, 0.71, 0.55}
\definecolor{upforestgreen}{rgb}{0.0, 0.27, 0.13}

% listing colors
\definecolor{listingBlue}{rgb}{0.13,0.13,1}
\definecolor{listingGreen}{rgb}{0,0.5,0}
\definecolor{listingRed}{rgb}{0.9,0,0}
\definecolor{listingGrey}{rgb}{0.46,0.45,0.48}

% listings ---------------------------------------------------------------------------------------------------- listings
\lstset{basicstyle=\ttfamily}
\lstset{literate=%
{Ö}{{\"O}}1
{Ä}{{\"A}}1
{Ü}{{\"U}}1
{ß}{{\ss}}1
{ü}{{\"u}}1
{ä}{{\"a}}1
{ö}{{\"o}}1
}
\lstset{language=Java,
showspaces=false,
showtabs=false,
breaklines=true,
showstringspaces=false,
breakatwhitespace=true,
commentstyle=\color{listingGreen},
keywordstyle=\color{listingBlue},
stringstyle=\color{listingRed},
basicstyle=\ttfamily,
moredelim=[il][\textcolor{listingGrey}]{$$},
moredelim=[is][\textcolor{listingGrey}]{\%\%}{\%\%}
}
\lstdefinelanguage{Kotlin}{
  keywords={package, as, typealias, this, super, val, var, fun, for, null, true, false, is, in, throw, return, break, continue, object, if, try, else, while, do, when, yield, typeof, yield, typeof, class, interface, enum, object, override, public, private, get, set, import, abstract, },
  keywordstyle=\color{listingBlue}\bfseries,
  ndkeywords={@Deprecated, Iterable, Int, Integer, Float, Double, String, Runnable, dynamic},
  ndkeywordstyle=\color{amethyst}\bfseries,
  emph={println, return@, forEach,},
  emphstyle={\color{orange}},
  identifierstyle=\color{black},
  sensitive=true,
  commentstyle=\color{listingGreen}\ttfamily,
  comment=[l]{//},
  morecomment=[s]{/*}{*/},
  stringstyle=\color{listingRed}\ttfamily,
  morestring=[b]",
  morestring=[s]{"""*}{*"""},
  numbers=left,
}

% misc ------------------------------------------------------------------------------------------------------------ misc
\graphicspath{ {../lectures-img/} }
\setlength{\parindent}{0pt}

\newcolumntype{C}[1]{>{\centering\arraybackslash}p{#1}} % zentriert mit Breitenangabe
\newcolumntype{L}[1]{>{\raggedright\arraybackslash}p{#1}} % rechtsbündig mit Breitenangabe
\newcolumntype{R}[1]{>{\raggedleft\arraybackslash}p{#1}} % rechtsbündig mit Breitenangabe


\definecolor{amethyst}{rgb}{0.6, 0.4, 0.8}
\definecolor{babyblue}{rgb}{0.54, 0.81, 0.94}
\definecolor{anti-flashwhite}{rgb}{0.95, 0.95, 0.96}
\definecolor{buff}{rgb}{0.94, 0.86, 0.51}
\tikzset{node blue/.style={circle,fill=babyblue!100,draw,minimum size=0.5cm,inner sep=0pt},}
\tikzset{node purp/.style={circle,fill=amethyst!100,draw,minimum size=0.5cm,inner sep=0pt},}
\tikzset{node whit/.style={circle,fill=anti-flashwhite!100,draw,minimum size=0.7cm,inner sep=0pt},}
\tikzset{node yell/.style={circle,fill=buff!100,draw,minimum size=0.7cm,inner sep=2pt},}
\begin{document}
    \header{Algorithmen und Berechenbarkeit}{Mitschrift}{Vorlesung 02}

    \section*{Min-Cut}
    \begin{definition}
        Ein \fett{Cut} unterteilt einen Graphen in zwei Hälften.
        Das heißt, es gibt Kanten, bei denen ein Knoten in einer
        Hälfte und bei denen der andere Knoten in der anderen Hälfte ist.
        Der \normal{Wert des Cuts} ist die \normal{Anzahl geschnittener Kanten}.
    \end{definition}

    \begin{figure}[htb]
        \centering

        \begin{tikzpicture}[thick, scale=0.5]
            \node[node purp] (n1) at (1,0) {\textcolor{white}{\sansserif{1}}};
            \node[node purp] (n2) at (0,3) {\textcolor{white}{\sansserif{2}}};
            \node[node blue] (n3) at (4,4) {\textcolor{black}{\sansserif{3}}};
            \node[node blue] (n4) at (7,2) {\textcolor{black}{\sansserif{4}}};
            \node[node blue] (n5) at (5,0) {\textcolor{black}{\sansserif{5}}};

            \foreach \from/\to in {n1/n2,n2/n3,n3/n4,n4/n5,n5/n1,n2/n5,n3/n5}
            \draw[very thick] (\from) -- (\to);

            \coordinate (c1) at (2,-1);
            \coordinate (c2) at (1.5,5);

            \draw [ultra thick, red, dashed] (c1) to [bend right=20] (c2);
        \end{tikzpicture}
        \caption*{Cut $\{1,2\}$ bzw.\ $\{3,4,5\}$ mit Wert 3}
    \end{figure}

    \begin{definition}
        Der \fett{Min-Cut} beschreibt den Cut, bei dem am wenigsten Kanten \normal{geschnitten} werden müssen,
        um den Graphen in zwei Hälften zu teilen. Ein Min-Cut muss nicht eindeutig sein.\\

        \normal{Formal:} Sei $G=(V,E)$ ein ungewichteter und ungerichteter Multigraph,
        $A \subset V$ und $cut(A, G)$ ein Cut.
        Finde $A \subset V$ mit $\vert cut(A,G)\vert$ minimal.
    \end{definition}

    \begin{figure}[H]
        \centering
        \begin{subfigure}{.5\textwidth}
            \centering
            \begin{tikzpicture}[thick, scale=0.5]
                \node[node purp] (n1) at (1,0) {\textcolor{white}{\sansserif{1}}};
                \node[node purp] (n2) at (0,3) {\textcolor{white}{\sansserif{2}}};
                \node[node purp] (n3) at (4,4) {\textcolor{white}{\sansserif{3}}};
                \node[node blue] (n4) at (7,2) {\textcolor{black}{\sansserif{4}}};
                \node[node purp] (n5) at (5,0) {\textcolor{white}{\sansserif{5}}};

                \foreach \from/\to in {n1/n2,n2/n3,n3/n4,n4/n5,n5/n1,n2/n5,n3/n5}
                \draw[very thick] (\from) -- (\to);


                \coordinate (c1) at (6.5,-1);
                \coordinate (c2) at (6,5);

                \draw [ultra thick, red, dashed] (c1) to [bend left=20] (c2);
            \end{tikzpicture}
        \end{subfigure}%
        \begin{subfigure}{.5\textwidth}
            \centering

            \begin{tikzpicture}[thick, scale=0.5]
                \node[node purp] (n1) at (1,0) {\textcolor{white}{\sansserif{1}}};
                \node[node blue] (n2) at (0,3) {\textcolor{black}{\sansserif{2}}};
                \node[node blue] (n3) at (4,4) {\textcolor{black}{\sansserif{3}}};
                \node[node blue] (n4) at (7,2) {\textcolor{black}{\sansserif{4}}};
                \node[node blue] (n5) at (5,0) {\textcolor{black}{\sansserif{5}}};

                \foreach \from/\to in {n1/n2,n2/n3,n3/n4,n4/n5,n5/n1,n2/n5,n3/n5}
                \draw[very thick] (\from) -- (\to);


                \coordinate (c1) at (3.5,-1);
                \coordinate (c2) at (-1.5,2);

                \draw [ultra thick, red, dashed] (c1) to [bend right=35] (c2);
            \end{tikzpicture}
        \end{subfigure}
        \caption*{Min-Cut links $\{4\}$ bzw.\ rechts $\{1\}$ mit Wert 2}
    \end{figure}

    \section*{Min-Cut-Algorithmus von Karger}
    \label{sec:kargersMincutAlgorithmus}

    Kargers-Min-Cut-Algorithmus beschreibt einen randomisierten Algorithmus, der mit einer
    Wahrscheinlichkeit von $\frac{1}{n^2}$ das richtige Ergebnis berechnet.
    Die zentrale Operation des Algorithmus ist die \kursiv{Kantenkontraktion}.
    Der Algorithmus lässt sich wie folgt beschreiben:

    \begin{figure}[H]
        \begin{algorithmic}
            \FOR{$i=1$ \TO $n-2$}
            \STATE{Kontrahiere zufällige Kante}
            \ENDFOR
            \STATE
            \RETURN{Kantenmenge zwischen den beiden verbleibenden Knoten}
        \end{algorithmic}
    \end{figure}

    \begin{figure}
        \centering
        \begin{subfigure}{.45\textwidth}
            \centering
            \begin{tikzpicture}[thick, scale=0.4]
                \node[node whit] (n1) at (-1,0) {};
                \node[node whit] (n2) at (4,3) {};
                \node[node whit] (n3) at (9,0) {};
                \node[node whit] (n4) at (4,-3) {};
                \node[node yell] (n5) at (2,0) {\sansserif{A}};
                \node[node yell] (n6) at (6,0) {\sansserif{B}};

                \foreach \from/\to in {n1/n5,n2/n5,n2/n6,n3/n6,n4/n5,n4/n6,n5/n6}
                \draw[very thick] (\from) -- (\to);
            \end{tikzpicture}
        \end{subfigure}%
        \begin{subfigure}{.1\textwidth}
            \centering
            $\Rightarrow$
        \end{subfigure}%
        \begin{subfigure}{.45\textwidth}
            \centering

            \begin{tikzpicture}[thick, scale=0.4]
                \node[node whit] (n1) at (-1,0) {};
                \node[node whit] (n2) at (4,3) {};
                \node[node whit] (n3) at (9,0) {};
                \node[node whit] (n4) at (4,-3) {};
                \node[node yell] (n5) at (4,0) {\sansserif{A, B}};

                \foreach \from/\to in {n1/n5,n3/n5}
                \draw[very thick] (\from) -- (\to);

                \draw [very thick] (n2) to [bend right=20] (n5);
                \draw [very thick] (n2) to [bend left=20] (n5);
                \draw [very thick] (n4) to [bend right=20] (n5);
                \draw [very thick] (n4) to [bend left=20] (n5);
            \end{tikzpicture}
        \end{subfigure}
        \caption*{Kantenkontraktion der Kante zwischen den Knoten \sansserif{A} und \sansserif{B}}
    \end{figure}

    Nach dem Ausführen des Algorithmus bleiben genau zwei Knoten übrig.
    Der Min-Cut ist die Anzahl der Kanten zwischen diesen Knoten.

    \subsection*{Min-Cut-Algorithmus von Karger: Analyse}

    \begin{itemize}
        \item \fett{Annahme:} Der Min-Cut ist eindeutig.
        \item {\fett{Beobachtung:}
        \begin{enumerate}
            \item Der Algorithmus berechnet genau dann das richtige Ergebnis,
            wenn er nie eine der Min-Cut-Kanten kontrahiert.
            \item Sei $k$ die Größe des Min-Cuts und $m$ die Anzahl der Kanten.
            Die Wahrscheinlichkeit im ersten Kontraktionsschritt einen Fehler zu machen, beträgt $\frac{k}{m}$,
            die Wahrscheinlichkeit keinen Fehler zu machen $1-\frac{k}{m}$.
        \end{enumerate} }
    \end{itemize}

    \begin{satz}
        Betrachte einen Multigraphen $G(V,E)$ mit einem Min-Cut-Wert $k$.
        Dann hat G mindestens $\frac{k \cdot n}{2}$ Kanten.
    \end{satz}
    \begin{proof}
        Sei $d_G(v)$ der Knotengrad von $v$. Jeder Knoten hat $d_G(v) \geq k$.
        Die Anzahl der Kanten ergibt sich somit zu
        \begin{equation*}
            \frac{\sum_{}^{}(d_G(v))}{2} \geq \frac{k \cdot n}{2}
        \end{equation*}
    \end{proof}\\

    Sei $E_i$ das Ereignis, dass im $i$-ten Kontraktionsschritt keine Min-Cut-Kante erwischt wurde.
    Für $m \geq \frac{k \cdot n}{2}$ beträgt die Wahrscheinlichkeit wie bekannt
    \begin{equation*}
        1-\frac{k}{m} \geq 1 - \frac{2}{n}
    \end{equation*}


    Falls $E_1$ eingetroffen ist, existieren vor dem zweiten Schritt mindestens $\frac{k \cdot (n-1)}{2}$ Kanten.
    Sei $\Pr(E_2 | E_1)$ die Wahrscheinlichkeit,
    im zweiten Schritt keinen Fehler zu machen, falls $E_1$ eingetreten ist.
    $\Pr(E_2 | E_1)$ beträgt
    \begin{equation*}
        1 - \frac{k}{m} \geq 1 - \frac{2}{n-1}
    \end{equation*}

    Falls $E_1, E_2, \dots , E_i$ eingetroffen sind,
    existieren vor dem $i+1$-Schritt mindestens $\frac{k \cdot (n-i)}{2}$ Kanten.
    Sei $\Pr(\bigwedge_{1}^i E_i|E_{i+1})$ die Wahrscheinlichkeit,
    im $i+1$-Schritt keinen Fehler zu machen, falls $E_1 \dots E_i$ eingetreten ist.
    $\Pr(\bigwedge_{1}^i E_i|E_{i+1})$  beträgt
    \begin{equation*}
        1 - \frac{k}{m} \geq 1 - \frac{2}{n-i}
    \end{equation*}

    Sei $\Pr_{correct}(\bigcap_{i+1}^{n-2})$ die Wahrscheinlichkeit, dass der Algorithmus korrekt arbeitet.
    Es gilt:
    \begin{align*}
        \Pr_{correct}(\bigcap_{i+1}^{n-2}) & =
        \Pr(E_1)\ \cdot\
        \Pr(E_2|E_1)\ \cdot\
        \Pr(E_3| E_1 \land E_2)\ \cdot\ \dots\ \cdot\
        \Pr(\bigwedge_{1}^i E_i|E_{i+1}) \\
        & \geq \prod_{i=1}^{n-2}1-\frac{2}{n - (i+1)} \\
        & = \prod_{i=1}^{n-2}\frac{n-i+1-2}{n-i+1} \\
        & = \prod_{i=1}^{n-2}\frac{n-i-1}{n-i+1} \\
        & = \frac{\cancel{n-2}}{n}\ \cdot\
        \frac{\cancel{n-3}}{n-1}\ \cdot\
        \frac{n-4}{\cancel{n-2}}\ \cdot\
        \frac{n-5}{\cancel{n-3}} \quad\dots \\
        & = \frac{2}{n \cdot (n-1)} \\
        & \\
        \Pr_{correct}(\bigcap_{i+1}^{n-2}) & \geq \frac{2}{n \cdot (n-1)}
    \end{align*}

    Die Erfolgswahrscheinlichkeit kann beliebig erhöht werden.
    Dafür muss der Algorithmus so oft wie gewünscht wiederholt werden.
    Der kleinste Wert, der in Folge dieser Wiederholungen herauskommt, ist der Min-Cut.\\

    Sei $r$ die Anzahl der Versuche.
    Die Wahrscheinlichkeit, \fett{nicht} den Min-Cut zu finden, ist
    \begin{align*}
        & \leq (1 - \frac{2}{n \cdot (n-1)})^r \\
        & \leq e^{-\frac{2}{n \cdot (n-1)}} = e^{\frac{-2r}{n \cdot (n-1)}}
    \end{align*}

    Sei $r = \frac{n \cdot (n-1)}{2}$.
    Die Wahrscheinlichkeit, einen falschen Min-Cut zu erhalten, ist
    \begin{equation*}
        \leq \frac{1}{e} \approx 0,36
    \end{equation*}

    Sei $r \approx n^2$. Es stellt sich eine \fett{nahezu konstante} Erfolgswahrscheinlichkeit ein.\\

    Sei $r \approx n^2 \cdot n \cdot \log(n)$.
    Das Ergebnis ist mit sehr hoher Wahrscheinlichkeit korrekt.
    Die Wahrscheinlichkeit, jetzt noch einen falschen Min-Cut zu haben, ist
    \begin{equation*}
        \leq \frac{1}{n^2}
    \end{equation*}
\end{document}